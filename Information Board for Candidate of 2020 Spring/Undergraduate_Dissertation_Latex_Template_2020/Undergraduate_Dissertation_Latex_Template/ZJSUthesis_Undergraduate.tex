%%%%%%%%%%%%%%%%%%%%%%%%%%%%%%%%%%%%%%%%%%%%%%%%%%%%%%%%%%%%%%%%
%%%              本模板可以使用以下两种方式编译:               %%%
%%%                 PDFLaTeX       XeLaTeX                   %%%
%%%                请使用 TeXLive 2015 编译                   %%%
%%%                  浙江工商大学论文模板                     %%%
%%%       Created:2016/04/15 ver 0.5 by KarryZhao@ZJSU      %%%
%%%  Any problem or advice please mail to Karry@outlook.com  %%%
%%%     modified by Richard Xin Gu 2018/4/15                              %%%
%%%%%%%%%%%%%%%%%%%%%%%%%%%%%%%%%%%%%%%%%%%%%%%%%%%%%%%%%%%%%%%%

\documentclass{ZJSUthesis}
\begin{document}
%%%%%%%%%%%%%%%%%%%%%%%%%%%%%%%%%%%%%%%%%%%%%%%%%%%%%%%%%%%%%%%%
%%%                                                          %%%
%%%                    填写论文封面信息                       %%%
%%%                                                          %%%
%%%%%%%%%%%%%%%%%%%%%%%%%%%%%%%%%%%%%%%%%%%%%%%%%%%%%%%%%%%%%%%%
\begin{titlepage}
\begin{center}
\thispagestyle{empty} %取消当前页码
\includegraphics[height=2.72cm]{picture/ZJGU_C}

\vspace{19pt}

\yihao \textbf{Undergraduate Dissertation}

\vspace{25pt}

\includegraphics[height=4.27cm]{picture/ZJGU}
\vspace{5mm}
\end{center}
\renewcommand{\ULthickness}{0.1pt}
\ULdepth=3pt \songti\sanhao

\makebox[15mm]{\textbf{Title:}} {The Effects of Government Spending on Real Exchange Rates}

\vspace{10mm}
%
%\makebox[38mm][s]{} {\hfill{收入比和财政支出的关系分析}\hfill}
%\vspace{1mm}

\begin{center}

\makebox[30mm][s]{} \makebox[30mm][s]{\textbf{School}} \uline{\hfill{School of Finance}\hfill} \makebox[30mm][s]{}
\vspace{4mm}

\makebox[30mm][s]{} \makebox[30mm][s]{\textbf{Major}} \uline{\hfill{International Finance}\hfill} \makebox[30mm][s]{}
\vspace{4mm}

%\makebox[30mm][s]{} \makebox[25mm][s]{\textbf{班级}} \uline{\hfill{时间序列分析}\hfill} \makebox[30mm][s]{}
%\vspace{4mm}

\makebox[30mm][s]{} \makebox[30mm][s]{\textbf{Student ID}} \uline{\hfill{1301060134}\hfill} \makebox[30mm][s]{}
\vspace{4mm}

\makebox[30mm][s]{} \makebox[30mm][s]{\textbf{Name}} \uline{\hfill{Saul Brown}\hfill} \makebox[30mm][s]{}
\vspace{4mm}

\makebox[30mm][s]{} \makebox[30mm][s]{\textbf{Supervisor}} \uline{\hfill{An Sangbei}\hfill} \makebox[30mm][s]{}
\vspace{4mm}

\makebox[30mm][s]{} \makebox[30mm][s]{\textbf{Submission}} \uline{\hfill{16 April, 2018}\hfill} \makebox[30mm][s]{}
\vspace{4mm}

\vspace{9mm}


\end{center}
\end{titlepage}


%%%%%%%%%%%%%%%%%%%%%%%%%%%%%%%%%%%%%%%%%%%%%%%%%%%%%%%%%%%%%%%%
%%%                                                          %%%
%%%                  中文摘要  英文摘要                       %%%
%%%                                                          %%%
%%%%%%%%%%%%%%%%%%%%%%%%%%%%%%%%%%%%%%%%%%%%%%%%%%%%%%%%%%%%%%%%
\newpage
\setlength{\baselineskip}{20pt}
\pagenumbering{roman}
\setcounter{page}{1}

\headheight 1.5cm % 页眉高
\pagestyle{fancy}
\lhead{School of Finance}
\rhead{Undergraduate Dissertation}



\title{中文摘要} % Title in Chinese
\date{}
\maketitle
%\cnabstract
\xiaosihao
本文采用OCED中34国的数据,构建了一个计量模型,检验政府支出对实际汇率的影响。$\cdots$ % input your Chinese Abstract here.


\cnkeywords{政府支出;~ 实际汇率;~ 面板数据} % insert three- five Keywords in Chinese

\newpage
\title{Abstract} % Title in English
\maketitle

%\enabstract
%\xiaosihao
% input your English Abstract below
Using panel data on military spending for 125 countries, we document new facts about the
effects of changes in government purchases on the real exchange rate, consumption, and current
accounts in both advanced and developing countries. While an increase in government
purchases causes real exchange rates to appreciate and increases consumption significantly in
developing countries, it causes real exchange rates to depreciate and decreases consumption in
advanced countries. The current account deteriorates in both groups of countries. These findings
are not consistent with standard international business-cycle models. We investigate whether the
difference between advanced economies and developing countries in the responses of real
exchange rates to spending shocks can be explained by alternative hypotheses.


\enkeywords{\wuhao Government Expenditure; ~ Real Exchange Rate; ~ Panel Data} % insert three - five keywords in English with first letter uppercase
\xiaosihao

\newpage
\pagenumbering{roman}
\setcounter{page}{1}
\renewcommand\contentsname{Table of Contents}

\tableofcontents
\renewcommand\tablename{Table}
\renewcommand\figurename{Figure}
\urlstyle{same}
%%%%%%%%%%%%%%%%%%%%%%%%%%%%%%%%%%%%%%%%%%%%%%%%%%%%%%%%%%%%%%%%
%%%                                                          %%%
%%%                        正文的开始                         %%%
%%%                                                          %%%
%%%%%%%%%%%%%%%%%%%%%%%%%%%%%%%%%%%%%%%%%%%%%%%%%%%%%%%%%%%%%%%%
\newpage
\pagenumbering{arabic}
\setcounter{page}{1}



%\chead{\line(1,0){435}}

%\chapter{浙江工商大学}
\section{Introduction}

How does government spending affect the current account and the real exchange rate? Conventional
wisdom—as well as mainstream macroeconomic models used by policymakers—suggests that
an increase in government spending puts pressure on the domestic currency to appreciate, leading
to current account deterioration (and potentially a “twin deficit”) and to a decrease in consumption
through an international risk-sharing condition. This mechanism holds across a wide range
of models, including both New Keynesian and neoclassical models. However, empirical evidence
for such a mechanism has not been settled. For example, Corsetti and Müller (2006) and Kim and
Roubini (2008) find that in the U.S. data the trade balance improves after a government spending
shock. In contrast, using the data for Australia, Canada, the United Kingdom, and the United States,
Monacelli and Perotti (2010) and Ravn, Schmitt-Grohé, and Uribe (2012) estimate that a rise in
government spending causes a trade deficit, as well as a real depreciation of the domestic currency
and an increase in consumption. Given these contrasting empirical results in studies of a relatively
small number of countries, several questions on the effects of government spending in an open
economy remain: First, does government spending cause the domestic currency to appreciate in
real terms and does it worsen the current account? Second, do the effects of government spending
shocks differ across countries, especially between advanced and developing countries? Third, are
there any other country characteristics, such as the exchange-rate regime or the degree of openness
to trade, that can affect the transmission mechanism of government spending shocks?
This paper addresses these important questions using a large data set for 125 countries between
1989 and 2013. We provide new evidence on the effects of government spending on the real exchange
rate, current account, and consumption. Importantly, we exploit the information in both
advanced and developing countries to distinguish between the effects of government spending in
these two groups. Our data also let us examine the differential effects of government spending
depending on exchange-rate regimes and the level of trade openness. Since government spending
can affect the state of the economy and vice versa, we identify government spending shocks using
exogenous variation in international military spending. This approach has been used in the closed economy
literature (Hall 2009, Barro and Redlick 2011, Ramey 2011), but remains under utilized
in the open-economy literature.
\subsection{Research Background}
We document a number of new empirical facts: First, in response to a positive government
spending shock, the real exchange rate appreciates on impact, and the effect is significant up to a
two-year horizon. After an increase in government spending of 1 percent of GDP, the real exchange
rate appreciates by over 3 percent on impact and by up to 5 percent two years after the shock.
The effect is most pronounced in countries with a flexible exchange-rate regime. Consistent with
Monacelli and Perotti (2010), we also find that the current account deteriorates significantly in
response to a positive government spending shock. Consumption increases substantially, peaking
at about 5 percent two years after the change in government spending.


\section{Literature Review}
A literature review is a text of a scholarly paper, which includes the current knowledge including substantive findings, as well as theoretical and methodological contributions to a particular topic. Literature reviews are secondary sources, and do not report new or original experimental work. Most often associated with academic-oriented literature, such reviews are found in academic journals, and are not to be confused with book reviews that may also appear in the same publication. Literature reviews are a basis for research in nearly every academic field\footnote{This part is borrowed from Wikipedia.}. A narrow-scope literature review may be included as part of a peer-reviewed journal article presenting new research, serving to situate the current study within the body of the relevant literature and to provide context for the reader. In such a case, the review usually precedes the methodology and results sections of the work.

\subsection{Empirical Research}
Producing a literature review may also be part of graduate and post-graduate student work, including in the preparation of a thesis, dissertation, or a journal article. Literature reviews are also common in a research proposal or prospectus (the document that is approved before a student formally begins a dissertation or thesis).

\subsection{Theoretical Research}

\section{Model Economy Setup}

\subsection{Households}
The household intends to maximize the following objective utility throughout her lifetime.
\begin{equation}\label{household utility}
  U_t = E_t\sum_{t=0}^{\infty}\beta^t\log C_t  - \varphi_t\frac{L_t^{1+\phi}}{\phi}\,,
\end{equation}
where $E_t$ is mathematical expectation operator, $C_t$ is consumption bundle. The objective utility \eqref{household utility} is subject to the following constraint.
\begin{equation}\label{budget constraint}
  P_tC_t \leq W_tL_t + Tr_t\,,
\end{equation}

\subsection{Producers}

\section{Econometric Results}
This is how you represent econometric results.

\subsection{Result Report}
\begin{table}[h] \caption{Performance Using Hard Decision Detection} %title of the table
\centering     % centering table
\begin{tabular}{c rrrrrrr}  % creating eight columns
\hline%\hline                        %inserting double-line
Audio Name&\multicolumn{7}{c}{Sum of Extracted Bits} \\ [0.5ex]    \hline                % inserts single-line
Police   & 5 & -1 &  5& 5& -7& -5& 3\\  % Entering row contents
Midnight & 7 & -3 &  5& 3& -1& -3& 5\\ News     & 9 & -3 &  7& 9& -5& -1& 9\\[1ex] % [1ex] adds vertical space
\hline                          % inserts single-line
\end{tabular} \label{tab:hresult} \end{table}

This section examines important cases that can affect our baseline results. In particular, we analyze
whether wars, financial crises, commodity prices, and the type of military spending can significantly
influence our baseline results. We also show that our results are robust to adding potentially
important controls to the regression.

\section{Figure}

This is how to insert picture.
\begin{figure}[!h]
  \begin{center}
  \centering
  \includegraphics[width= 6cm]{picture/ZJSU_logo}
  \caption{Zhejiang Gongshang University Old Logo}\label{zlog}
  \end{center}
\end{figure}


 Language is an interesting subject \citep{TM83}.  There are commands for in-text citations, like \citet{GMP81}. And you can pass an option to specify additional details, such as a page or chapter number, as an option \citep[p. 130]{Ful83}.

   A very important reference in financial accelerator is \citep{BGG99} and \citet*{BGG99}.





\newpage
\renewcommand\refname{Reference}
%\begin{thebibliography}{10}
%\bibitem{1} \url{http://www.mcm.edu.cn/}
%\bibitem{1} \url{http://bbs.chinatex.org}
%\bibitem{2} \url{http://www.chinatex.org}
%\bibitem{3} Alpha Huang, \textbf{latex-notes-zh-cn}, 2014.
%\bibitem{lf}M.R.C. van Dongen,\textbf{\LaTeX-and-Friends}, 2013.
%\bibitem{figure}Keith Reckdahl,\textbf{Using Import graphics in \LaTeXe}, 1997.
%\bibitem{HM}Addison Wesley,\textbf{Higher Mathematics}, 下载地址如下\\ \url{http://media.cism.it/attachments/ch8.pdf}
%\end{thebibliography}
\bibliographystyle{humanbio} % or try abbrvnat or unsrtnat
\makeatletter\renewcommand{\@biblabel}[1]{[#1].}\makeatother
\addcontentsline{toc}{section}{Reference}
\bibliography{example} % refers to example.bib

%\bibliographystyle{agsm}
%%\cleardoublepage
%\normalbaselines %Fixes spacing of bibliography
%\addcontentsline{toc}{chapter}{Bibliography} %adds Bibliography to your table of contents
%\bibliography{refer}



\newpage
\appendix
\section*{Appendix}

\newpage
\end{document}
